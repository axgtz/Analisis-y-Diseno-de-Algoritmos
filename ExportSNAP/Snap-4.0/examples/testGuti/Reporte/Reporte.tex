\documentclass[prodmode,acmtecs]{acmsmall} % Aptara syntax

\usepackage[ruled]{algorithm2e}
\usepackage{cite}
\usepackage{listings,multicol}
\usepackage{hyperref}

\renewcommand{\algorithmcfname}{ALGORITHM}
\SetAlFnt{\small}
\SetAlCapFnt{\small}
\SetAlCapNameFnt{\small}
\SetAlCapHSkip{0pt}
\IncMargin{-\parindent}

% Metadata Information
\acmVolume{1}
\acmNumber{1}
\acmYear{2017}
\acmMonth{10}

%\issn{1234-56789}

\begin{document}

\title{Exportar Grafos en formatos: GDF, GraphSON, GraphML , GEXF}

\author{
Roberto Alejandro Guti\'{e}rrez Guill\'{e}n\\
Instituto Tecnol�gico y de Estudios Superiores de Monterrey, Campus Santa Fe\\
Mexico\\
A01019608@itesm.mx
       }

\maketitle

\begin{abstract}
En este art\'{i}culo de investigac\'{i}on se demostraran las diferentes formas de exportar un grafo, usando la l\'{i}breria de SNAP para C++, los formatos a exportar el grafo son los siguientes: GDF, GraphSON, GraphML y GEXF. Adem\'{a}s se discutir\'{a}n los aspectos positivos y negativos de cada formato de exportaci\'{o}n.
\end{abstract}

\vspace{5mm} 

\begin{CCSXML}
<ccs2012>
<concept>
<concept_id>10003752.10003809.10003635</concept_id>
<concept_desc>Theory of computation~Graph algorithms analysis</concept_desc>
<concept_significance>500</concept_significance>
</concept>
</ccs2012>
\end{CCSXML}


\ccsdesc[500]{Theory of computation~Graph algorithms analysis}

 
\vspace{5mm} 
\keywords{Graphs, acm, export, c++, GraphSON, GEXF, GDF, GraphML }

\vspace{5mm} 
\acmformat{Roberto Alejandro Guti\'{e}rrez Guill\'{e}n. 2017. Exportar Grafos en formatos: GDF, GraphSON, GraphML , GEXF.}



\vspace{5mm} 

\section{Github}
https://github.com/axgtz/Analisis-y-Diseno-de-Algoritmos/tree/master/ExportSNAP/Snap-4.0/examples/testGuti
\vspace{5mm} 
\section{Introducci\'{o}n }
Los grafos son una estructura de datos de la computaci\'{o}n. Son un conjunto de nodos(v\'{e}rtices) que representan a un objeto, y estan unidos por aristas. Puede que las uniones sean dirigidas o no, tambi\'{e}n, pueden tener peso y otros atributos. Son usados para muchas cosas, por ejemplo para mostrar un mapa de amistades en Facebook o una red de computadoras.



\section{Procedimiento}
Primeramente, se elige una dataset de la siguiente pagina https://snap.stanford.edu/data/index.html, si el numero de nodos y uniones es menor, sera mas facil de procesar. Preferiblemente elegir un dataset que sea dirigido, para obtener resultados m\'{a}s interesantes.

Despues, se baja SNAP de la siguiente pagina https://snap.stanford.edu, usando la documentaci\'{o}n proporcionada se compila todos los programas incluidos. En tercer lugar, se usa un proyecto de la secci\'{o}n de examples y se modifica para poder leer el dataset que se descargo y tambi\'{e}n para poder exportar en los cuatro formatos especificados(GDF, GraphSON, GraphML , GEXF). 

En la secci\'{o}n de lectura se declara el grafo el cual en este caso es dirigido. Despu\'{e}s, se agregan los nodos al grafo y luego todas las uniones. Al acabar se miden los tiempos al ejecutar cada forma de exportar. Al finalizar de correr el c\'{o}digo, se pueden visualizar los grafos exportados usando el programa Gephi. 

\section{Ventajas y Desventajas}
\subsection{GDF}
Este formato esta basado en el formato CSV(Comma Separated File), por lo tanto es muy f'{a}cil de leer y de convertir a CSV. Adme'{a}s fue el formato m\'{a}s r'\{a}pido de exportar por su simple estructura.

\subsection{GEXF}
Este formato es muy usado en la industria, por lo tanto es muy maduro y robusto, por lo que da confianza. Es muy parecido a GraphML ya que tambi\'{e}n se basa en XML pero es menos utilizado en la industria.

\subsection{GraphML}
GraphML es un formato para almacenar datos basado en XML, y soporta menos informaci�n adicional que sus rivales, esto ayuda en su facilidad de exportar pero no se pueden incluir grafos complejos.

\subsection{GraphSON}
Una gran desventaja es que la \'{u}ltima versi\'{o}n de Gephi, una aplicaci\'{o}n usada para visualizar grafos, no soporta este formato en su \'{o}ltima versi'\{o}n. Es el formato m'{a}s complejo de exportar porque no existe un est\'{a}ndar pero esto tiene la ventaja de que puede ser adaptado a cualquier situaci\'{o}n. 
\vspace{5mm} 


\section{Complejidades Temporal y Espacial}
\subsection{Complejidad Temporal }
Los cuatro formatos de exportaci'{o}n  tienen la misma complejidad temporal, ya que guardan todos los nodos y todos los vertices, y se tienen que recorrer todos para ser exportados, por lo tanto complejidad temporal de los cuatro algoritmos es de O(V+E).\subsection{Complejidad Espacial }
La complejidad espacial es O(V+E), la suma de la cantidad de nodos y uniones.


\section{Tiempos de ejecuci\'{o}n}
\begin{tabular}{| p{6.2cm} | p{5.2cm} | }
\hline
\\
Tiempo de ejecuci\'{o}n GDF: & 0.0369601 segundos	\\

Tiempo de ejecuci\'{o}n GEXF: & 0.0503775 segundos	\\

Tiempo de ejecuci\'{o}n GraphML: & 0.0511828 segundos\\

Tiempo de ejecuci\'{o}n GraphSON: & 0.0392701 segundos\\
\end{tabular}
\vspace{5mm} 





\section{Gephi}        

\begin{figure}
\centering
\includegraphics[width=18cm]{grafo}
\caption{Resultado de visualizaci�n en Gephi}
\end{figure}
                             
       


\section{Apendice} 

\bibliography{citations}
\appendix
\section{Funci\'{o}n para exportar a GDF}
\begin{lstlisting}

void exportarGDF(PNGraph Graph){
    ofstream archivo_salida;
    archivo_salida.open("grafoGnutella.gdf");
    
    if (archivo_salida.is_open()){
        archivo_salida << "nodedef>name\n";
        for (TNGraph::TNodeI NI = Graph->BegNI(); NI < Graph->EndNI(); NI++){
            archivo_salida << NI.GetId() << ",\n";
        }
        
        archivo_salida << "edgedef>node1,node2\n";
        for (TNGraph::TEdgeI EI = Graph->BegEI(); EI < Graph->EndEI(); EI++){
            archivo_salida << EI.GetSrcNId() << "," << EI.GetDstNId() << "\n";
        }
        
        archivo_salida.close();
    }
}
\end{lstlisting}

\vspace{5mm} 

\section{Funci\'{o}n para exportar a GraphSON}
\begin{lstlisting}

void exportarGrapSON(PNGraph Graph){
    ofstream archivo_salida;
    archivo_salida.open("grafoGnutella.json");
    
    if (archivo_salida.is_open()){
        archivo_salida << "{ \"graph\": {\n";
        archivo_salida << "\"nodes\": [\n";
        
        for (TNGraph::TNodeI NI = Graph->BegNI(); NI < Graph->EndNI(); NI++){
            archivo_salida << "{\n \"_id\": \"" << NI.GetId() << "\" }";
            
            if (NI == Graph->EndNI()){//Si es el ultimo nodo
                archivo_salida << " ],\n";
            }else{//Todos los demas nodos
                archivo_salida << ",\n";
            }
        }
        
        archivo_salida << "\"edges\": [\n";
        for (TNGraph::TEdgeI EI = Graph->BegEI(); EI < Graph->EndEI(); EI++){
            archivo_salida << "{ \"source\": \"" << EI.GetSrcNId() << "\", \"target\": \"" << EI.GetDstNId() << "\" }";
            
            if (EI == Graph->EndEI()){
                archivo_salida << " ]\n";
            }else{
                archivo_salida << ",\n";
            }
        }
        archivo_salida << "} }";
        
    }
}
\end{lstlisting}

\vspace{5mm} 

\section{Funci\'{o}n para exportar GraphML}
\begin{lstlisting}

void exportarGraphMl(PNGraph Graph){
    ofstream archivo_salida;
    archivo_salida.open("grafoGnutella.graphml");
    
    if (archivo_salida.is_open()){
        archivo_salida << "<?xml version=\"1.0\" encoding=\"UTF-8\"?>\n";
        archivo_salida << "<graphml xmlns=\"http://graphml.graphdrawing.org/xmlns\" xmlns:xsi=\"http://www.w3.org/2001/XMLSchema-instance\" xsi:schemaLocation=\"http://graphml.graphdrawing.org/xmlns http://graphml.graphdrawing.org/xmlns/1.0/graphml.xsd\">\n";
        archivo_salida << "<graph id=\"G\" edgedefault=\"directed\">\n";
        
        for (TNGraph::TNodeI NI = Graph->BegNI(); NI < Graph->EndNI(); NI++){
            archivo_salida << "<node id=\"n" << NI.GetId() << "\"/>\n";
        }
        int i = 0;
        for (TNGraph::TEdgeI EI = Graph->BegEI(); EI < Graph->EndEI(); EI++,i++){
            archivo_salida << "<edge id=\"e" << i << "\" source=\"n" << EI.GetSrcNId() << "\" target=\"n" << EI.GetDstNId() << "\" />\n";
        }
        
        archivo_salida << "</graph>\n";
        archivo_salida << "</graphml>\n";
        archivo_salida.close();
    }
}
\end{lstlisting}

\vspace{5mm} 

\section{Funci\'{o}n para exportar a GEXF}
\begin{lstlisting}
void exportarGEXF(PNGraph Graph){//
    ofstream archivo_salida;
    archivo_salida.open("grafoGnutella.gexf");
    
    if (archivo_salida.is_open()){
        archivo_salida << "<?xml version=\"1.0\" encoding=\"UTF-8\"?>\n";
        archivo_salida << "<gexf xmlns=\"http://www.gexf.net/1.2draft\" version=\"1.2\">\n";
        archivo_salida << "<graph mode=\"static\" defaultedgetype=\"directed\">\n";
        archivo_salida << "<nodes>\n";
        
        
        for (TNGraph::TNodeI NI = Graph->BegNI(); NI < Graph->EndNI(); NI++){
            archivo_salida << "<node id=\"" << NI.GetId() << "\" />\n";
        }
        
        archivo_salida << "</nodes>\n";
        archivo_salida << "<edges>\n";
        int i = 0;
        for (TNGraph::TEdgeI EI = Graph->BegEI(); EI < Graph->EndEI(); EI++,i++){
            archivo_salida << "<edge id=\"" << i << "\" source=\"" << EI.GetSrcNId() << "\" target=\"" << EI.GetDstNId() << "\" />\n";
        }
        archivo_salida << "</edges>\n";
        archivo_salida << "</graph>\n";
        archivo_salida << "</gexf>\n";
        archivo_salida.close();
    }
}

\end{lstlisting}


%\nocite{*}
%\bibliographystyle{acm}
%\bibliography{citatarea}

\begin{thebibliography}{9}

\bibitem{1}
  Stanford. 2002. Gnutella peer-to-peer network: August 4 2002.  Retrieved October,2017 from https://snap.stanford.edu/data/p2p-Gnutella04.html. 
  
  \bibitem{2}
  Dan LaRocque. 2014 . GraphSON Format. Retrieved October,2017 from https://github.com/thinkaurelius/faunus/wiki/GraphSON-Format    
  
  \bibitem{3}
  Charles Iliya Krempeaux. 2013. GDF: A CSV Like Format For Graphs. Retrieved October,2017 from http://changelog.ca/log/2013/03/09/gdf. 
  
     \bibitem{4}
 Gephi. 2017. GraphML Format. Retrieved October, 2017 from https://gephi.org/users/supported-graph-formats/graphml-format/
 
	\bibitem{5}
 Gephi. 2017. GEXF File Formatt. Retrieved October, 2017 from https://gephi.org/gexf/format/

\end{thebibliography}
\end{document}